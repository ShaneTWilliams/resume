\documentclass[letterpaper,11pt]{article}

\usepackage{latexsym}
\usepackage[empty]{fullpage}
\usepackage{titlesec}
\usepackage{marvosym}
\usepackage[usenames,dvipsnames]{color}
\usepackage{verbatim}
\usepackage{enumitem}
\usepackage[hidelinks]{hyperref}
\usepackage{fancyhdr}
\usepackage[english]{babel}
\usepackage{tabularx}
\input{glyphtounicode}

\pagestyle{fancy}
\fancyhf{} % clear all header and footer fields
\fancyfoot{}
\renewcommand{\headrulewidth}{0pt}
\renewcommand{\footrulewidth}{0pt}

% Adjust margins
\addtolength{\oddsidemargin}{-0.5in}
\addtolength{\evensidemargin}{-0.5in}
\addtolength{\textwidth}{1in}
\addtolength{\topmargin}{-.5in}
\addtolength{\textheight}{1.0in}

\urlstyle{same}

\raggedbottom
\raggedright
\setlength{\tabcolsep}{0in}

% Sections formatting
\titleformat{\section}{
  \vspace{-4pt}\scshape\raggedright\large
}{}{0em}{}[\color{black}\titlerule \vspace{-5pt}]

% Ensure that generate pdf is machine readable/ATS parsable
\pdfgentounicode=1


\newcommand{\resumeItem}[1]{
  \item\small{
    #1 \vspace{-2pt}
  }
}
\newcommand{\resumeHeading}[4]{
    \begin{tabular*}{0.99\textwidth}[t]{l@{\extracolsep{\fill}}r}
        \textbf{#1} & #2 \\
        \textit{\small#3} & \textit{\small #4} \\
    \end{tabular*}\vspace{-5pt}
}
\newcommand{\resumeSubheading}[4]{
    \vspace{-1pt}\item[]
    \begin{tabular*}{0.97\textwidth}[t]{l@{\extracolsep{\fill}}r}
        \textbf{#1} & #2 \\
        \textit{\small#3} & \textit{\small #4} \\
    \end{tabular*}\vspace{-5pt}
}
\newcommand{\projectSubheading}[2]{
    \vspace{-1pt}\item[]
    \begin{tabular*}{0.97\textwidth}[t]{l@{\extracolsep{\fill}}r}
        \textbf{#1} & \textit{\small #2} \\
    \end{tabular*}\vspace{-5pt}
}
\newcommand{\resumeSubSubheading}[2]{
    \begin{tabular*}{0.97\textwidth}{l@{\extracolsep{\fill}}r}
        \textit{\small#1} & \textit{\small #2} \\
    \end{tabular*}\vspace{-5pt}
}
\newcommand{\resumeSubItem}[1]{\resumeItem{#1}\vspace{-4pt}}
\newcommand{\resumeSubHeadingListStart}{\begin{itemize}[leftmargin=*]}
\newcommand{\resumeSubHeadingListEnd}{\end{itemize}}
\newcommand{\resumeItemListStart}{\begin{itemize}}
\newcommand{\resumeItemListEnd}{\end{itemize}\vspace{-2pt}}
\newcommand{\skillItem}[2]{
    \item[] {\textbf{#1}: #2}
}

\renewcommand{\labelitemii}{$\circ$}


\begin{document}

% HEADING %
\begin{tabular*}{\textwidth}{l@{\extracolsep{\fill}}r}
  \textbf{\LARGE Shane Williams} & \href{https://www.linkedin.com/in/shanetwilliams/}{linkedin.com/in/shanetwilliams} \\
  \href{mailto:shane.williams@mun.ca}{shane.williams@mun.ca} & \href{http://github.com/shanetwilliams/}{github.com/shanetwilliams} \\
\end{tabular*}


% EDUCATION %
\section{Education}
\resumeSubHeadingListStart
    \resumeSubheading
        {Memorial University of Newfoundland}{St. John's, Canada}
        {Bachelor of Computer Engineering; GPA: 4.00 / 95.2\%}{2019 -- 2024}
\resumeSubHeadingListEnd


% EXPERIENCE %
\section{Experience}
\resumeSubHeadingListStart

    \resumeSubheading
        {Tesla}{Palo Alto, CA}
        {Firmware Platforms Intern}{Fall 2023}
    \resumeItemListStart
    \resumeItem{Developed low-level firmware features for use on multi-core, multi-MCU vehicle controllers with an emphasis on firmware developer experience.}
    \resumeItem{Brought up new multi-MCU PCBs for use in upcoming vehicle programs.}
    \resumeItem{Developed firmware tooling in Python to automate initial board bringup and the generation of board-level drivers.}
    \resumeItemListEnd

    \resumeSubSubheading
    {Firmware Integration Intern}{Summer 2022}
    \resumeItemListStart
        \resumeItem{Owned several firmware features, working with external suppliers and mechanical, electrical, firmware, regulatory and safety teams across Tesla to bring them into production.}
        \resumeItem{Wrote safety-critical user-facing vehicle firmware in C for Tesla vehicle interiors.}
        \resumeItem{Presented prototype vehicle features to company executives.}
    \resumeItemListEnd

    \resumeSubheading
        {NVIDIA}{Santa Clara, CA}
        {Firmware Development Intern}{Fall 2021}
    \resumeItemListStart
        \resumeItem{Wrote U-Boot and embedded Linux firmware for an ARM Cortex-A processor on a PCIe device.}
        \resumeItem{Developed device drivers in C for a PCIe controller and PHY.}
        \resumeItem{Created kernel and userspace software for PCIe host and endpoint devices.}
    \resumeItemListEnd

    \resumeSubheading
        {Kepler Communications}{Toronto, Canada}
        {Firmware Development Intern}{Winter 2021}
    \resumeItemListStart
        \resumeItem{Owned firmware development for a new cutting-edge satellite radio using C on a Cortex-M platform.}
        \resumeItem{Reviewed, brought up and debugged this PCB in a remote environment using my home lab.}
        \resumeItem{Wrote satellite flight computer firmware using C and FreeRTOS on a Cortex-R processor in an SOC.}
    \resumeItemListEnd

    \resumeSubSubheading
    {Software Development Intern}{Summer 2020}
    \resumeItemListStart
        \resumeItem{Wrote a file system and ARQ protocol for satellite communications using C++ on a Linux platform, improving satellite throughput by 5-200\%.}
    \resumeItemListEnd

\resumeSubHeadingListEnd


% PROJECTS %
\section{Projects}
\resumeSubHeadingListStart

    \projectSubheading
        {Senior Capstone Project -- Remotely Operated Quadcopter}{2023 -- 2024}
    \resumeItemListStart
        \resumeItem{Designed one of two onboard drone PCBs, which included a microcontroller, digital radio, USB-PD sink, power regulation, GPS, video transmitter, and a Qi wireless charging sink.}
        \resumeItem{Wrote all onboard firmware, including PID controls, using FreeRTOS on a Cortex-M microcontroller.}
        \resumeItem{Created tooling in Python for generating peripheral drivers for the 10+ onboard ICs from YAML register maps.}
        \resumeItem{Wrote ground station software and a control terminal program using Python and React with gRPC.}
    \resumeItemListEnd

    \projectSubheading
        {Student Team -- Killick-1 Cubesat}{2020 -- 2022}
    \resumeItemListStart
        \resumeItem{Designed the software architecture for a scientific nanosatellite ground station and led a team of four students in its implementation.}
        \resumeItem{Wrote Python software for satellite tracking, communication and scheduling, telemetry database management, and data visualization.}
    \resumeItemListEnd

\resumeSubHeadingListEnd


% SKILLS %
\section{Skills}
\resumeSubHeadingListStart

    \vspace{6pt}
    \setlength\itemsep{0px}
    \skillItem{Languages}{C, C++, Python, Java, Rust, Javascript}
    \skillItem{Tools \& Frameworks}{Linux, U-Boot, FreeRTOS, Make, CMake, SCons, GDB, Git, Docker, OpenGL}
    \skillItem{Protocols}{HDMI, PCIe, TCP, UDP, SPI, I2C, I2S, UART, CAN, LIN, LoRa}
    \skillItem{Hardware}{PCB schematic \& layout with KiCAD; LV system design; PCB bringup, rework, and lab skills }

    \resumeSubHeadingListEnd

\end{document}
